\documentclass[11pt,a4paper]{article}
\usepackage[latin1]{inputenc}
\usepackage[T1]{fontenc}
\usepackage{a4,german}

\pagestyle{plain}
\voffset=-1in
\hoffset=-1in
\setlength{\oddsidemargin}{2cm}
\setlength{\textwidth}{17cm}
\setlength{\topmargin}{1.5cm}
\setlength{\textheight}{24cm}
\setlength{\parindent}{0pt}
\setlength{\parskip}{1ex}
\setlength{\fboxsep}{2.5mm}

\begin{document}
\section*{"Ubungen zum Standardmodell der Teilchenphysik}
\subsection*{5. Freier Propagator}
Ein freies, nicht-relativistisches, quantenmechanisches Teilchen der Masse $m$ bewege sich im 3-dimensio"-nalen Raum. Berechnen Sie die "Ubergangsamplitude vom Punkt $\vec{x}$ zur Zeit $t = 0$ zum Punkt $\vec{x}'$ zur Zeit $t = T$ mit dem Pfadintegral-Formalismus.\par
Vergleichen Sie das Resultat mit dem Fall, bei welchem nur der klassische Pfad ber"ucksichtigt wird.
\subsubsection*{L"osung}
Die "Ubergangsamplitude nach dem Pfadintegralformalismus ist (gem"a"s Vorlesung) wie folgt definiert:
\begin{eqnarray}
\langle\vec{x}'|\hat{U}(T,0)|\vec{x}\rangle = \int Dx \, \exp\left(\frac{i}{\hbar}\Delta\sum_{i=0}^{N-1}\left[\frac{m}{2}\left(\frac{\vec{x}_{i+1}-\vec{x}_i}{\Delta}\right)^2 - V(\vec{x}_i) \right]\right)
\end{eqnarray}
F"ur ein freies Teilchen ist das Potential $V(\vec{x}_i)=0$. Das Funktionalintegral $\int Dx$ ist definiert als:
\begin{eqnarray}
\int Dx = \lim_{\Delta\to0} \left( \frac{m}{2\pi i\hbar\Delta}\right)^{\frac{d}{2}N}\int d^dx_1 \, d^dx_2 \ldots d^dx_{N-1}
\end{eqnarray}
In unserem Beispiel ist $d=3$. Wir zerlegen den Pfad von $\vec{x} = \vec{x}_0$ nach $\vec{x}' = \vec{x}_N$ in $N$ Teilst"ucke. Im Limes betrachten wir den "Ubergang $\Delta\to0$. Mit $t_N - t_0 = T = N \cdot \Delta$ ergibt sich nat"urlich auch $N\to\infty$. Damit gilt es zu l"osen:
\begin{eqnarray}
\langle\vec{x}'|\hat{U}(T,0)|\vec{x}\rangle = \lim_{\Delta\to0} \left( \frac{m}{2\pi i\hbar\Delta}\right)^{\frac{3}{2}N}\int d^3x_1 \, d^3x_2 \ldots d^3x_{N-1} \exp\left(\frac{i}{\hbar}\frac{m}{2}\Delta \sum_{i=0}^{N-1}\left(\frac {\vec{x}_{i+1}-\vec{x}_i}{\Delta}\right)^2\right)
\end{eqnarray}
Das eine $\Delta$ im Exponenten k"onnen wir vor die Summe ziehen. Die verbleibenden Faktoren vor der Summe fassen wir zusammen:
\[C = \frac{i}{\hbar}\frac{m}{2\Delta}\]
Offensichtlich ist $\Re$e$(C)=0$. Bei den $N$ Integrationen sind jeweils nur einige Terme der Summation im Exponenten der Exponentialfunktion zu ber"ucksichtigen. F"ur die erste Integration $\int d^3x_1$ suchen wir daher die L"osung von 
\[ \int d^3x_1 \exp \left(C(\vec{x}_1-\vec{x}_0)^2 - C(\vec{x}_2 - \vec{x}_1)^2 \right)\]
An dieser Stelle m"ochten wir als mathematischen Einschub die allgemeine L"osung einer Formel dieser Struktur einf"ugen. Sie gilt unter der Bedingung $\Re$e$(\alpha)\le0$ und $\Re$e$(\beta)\le0$.
\begin{eqnarray}
\int\limits_{-\infty}^{+\infty} d^3y \; \exp \left[\alpha(x-y)^2 + \beta(z-y)^2 \right] = \left(\frac{-\pi}{\alpha  + \beta}\right)^\frac{3}{2} \exp \left[\frac{\alpha\beta}{\alpha + \beta}(z-x)^2 \right]
\end{eqnarray}
Die Bedingung ist offenbar erf"ullt. Damit l"a"st sich die erste Integration ausf"uhren:
\begin{eqnarray}
\int d^3x_1 \, \exp \left[C(\vec{x}_1-\vec{x}_0)^2 + C(\vec{x}_2 - \vec{x}_1)^2 \right] = \left(\frac{-\pi}{2C}\right)^\frac{3}{2} \exp \left[\frac{C}{2}(\vec{x}_2-\vec{x}_0)^2 \right]
\end{eqnarray}
\newpage
Auch die n"achste Integration m"ochten wir noch explizit ausf"uhren:
\begin{eqnarray}
\int d^3x_1 \, d^3x_2 \; e^{C(\vec{x}_1-\vec{x}_0)^2 + C(\vec{x}_2 - \vec{x}_1)^2 + C(\vec{x}_3 - \vec{x}_2)^2}  & = & \left(\frac{-\pi}{2C}\right)^\frac{3}{2} \int d^3x_2 \, \exp \left[\frac{C}{2}(\vec{x}_2-\vec{x}_0)^2 + C(\vec{x}_3 - \vec{x}_2)^2 \right]\nonumber\\ & = & \left(\frac{-\pi}{2C}\right)^\frac{3}{2} \left(\frac{-\pi}{\frac{3}{2}C}\right)^\frac{3}{2} \exp \left[\frac{\frac{1}{2}C^2}{\frac{3}{2}C}(\vec{x}_3-\vec{x}_0)^2 \right]\nonumber\\ & = & \left(\frac{(-\pi)^2}{3C^2}\right)^\frac{3}{2} \exp \left[\frac{C}{3}(\vec{x}_3-\vec{x}_0)^2 \right]
\end{eqnarray}
Der Vergleich von (6) mit (5) liefert den rekursiven Charakter aller $N-1$ Integrationen. Das Gesamt\-integral ergibt somit:
\begin{eqnarray}
\lefteqn{\int d^3x_1 d^3x_2 \ldots d^3x_{N-1} \exp\left(C\sum_{i=0}^{N-1}(\vec{x}_{i+1}-\vec{x}_i)^2\right) } \hspace{5cm} \nonumber\\ & = & \left( \frac{(-\pi)^{N-1}}{((N-1)+1)C^{N-1}} \right)^\frac{3}{2} \exp\left[\frac{C}{(N-1)+1}(\vec{x}_{(N-1)+1} - \vec{x}_0)^2\right]\nonumber\\& = & \left( \frac{(-\pi)^{N-1}}{NC^{N-1}} \right)^\frac{3}{2} \exp\left[\frac{C}{N}(\vec{x}_{N} - \vec{x}_0)^2\right]
\end{eqnarray}
Dieses Ergebnis k"onnen wir nun in (3) einsetzen:
\begin{eqnarray*}
\langle\vec{x}'|\hat{U}(T,0)|\vec{x}\rangle & = & \lim_{\Delta\to0} \left( \frac{m}{2\pi i\hbar\Delta}\right)^{\frac{3}{2}N}\int d^3x_1 \ldots d^3x_{N-1} \exp\left(C\sum_{i=0}^{N-1}(\vec{x}_{i+1}-\vec{x}_i)^2\right)\nonumber\\
& = & \lim_{\Delta\to0} \left( \frac{m}{2\pi i\hbar\Delta}\right)^{\frac{3}{2}N} \left( \frac{(-\pi)^{N-1}}{NC^{N-1}} \right)^\frac{3}{2} \exp\left[\frac{C}{N}(\vec{x}_N - \vec{x}_0)^2\right]\nonumber\\
& = & \lim_{\Delta\to0} \left( \frac{m}{2\pi i\hbar\Delta}\right)^{\frac{3}{2}} \left( \frac{m}{2\pi i\hbar\Delta}\right)^{\frac{3}{2}(N-1)} \left(\frac{1}{N}\right)^{\frac{3}{2}} \left( \frac{(-\pi)}{C} \right)^{\frac{3}{2}(N-1)} \exp\left[\frac{C}{N}(\vec{x}_N - \vec{x}_0)^2\right]\nonumber\\
& = & \lim_{\Delta\to0} \left( \frac{m}{2\pi i\hbar\Delta N}\right)^{\frac{3}{2}} \left( \frac{m}{2\pi i\hbar\Delta}\right)^{\frac{3}{2}(N-1)} \left( \frac{(-\pi)2\hbar\Delta}{im} \right)^{\frac{3}{2}(N-1)} \exp\left[\frac{C}{N}(\vec{x}_N - \vec{x}_0)^2\right]\\ & = & \lim_{\Delta\to0} \left( \frac{m}{2\pi i\hbar\Delta N}\right)^{\frac{3}{2}} \exp\left[\frac{C}{N}(\vec{x}_N - \vec{x}_0)^2\right]
\end{eqnarray*}
Nun setzen wir wieder die Definition von $C=\frac{i}{\hbar}\frac{m}{2\Delta}$ ein. Mit $T = \Delta N$ entf"allt auch die Limesbildung. Auch ersetzen wir $\vec{x}_N = \vec{x}'$ sowie $\vec{x}_0 = \vec{x}$ und erhalten als Endergebnis:
\begin{equation}
\fbox{$ \displaystyle
\langle\vec{x}'|\hat{U}(T,0)|\vec{x}\rangle =  \left( \frac{m}{2\pi i\hbar T}\right)^{\frac{3}{2}} \exp\left[\frac{i}{\hbar}\frac{m}{2T}(\vec{x}' - \vec{x})^2\right] $}
\end{equation}
Die klassische Betrachtung liefert die Wirkung:
\begin{eqnarray*}
S[x] = \int \! dt \, L(x,\dot{x}) = \int_0^T \! dt \, \frac{mv^2}{2} = T \frac{m}{2}v^2 = T \frac{m}{2}\frac{(x' - x)^2}{T^2} = \frac{m}{2}\frac{(x' - x)^2}{T}
\end{eqnarray*}
Der Pfadintegralformalismus liefert also exakt das gleiche Ergebnis wie die klassische Betrachtung.
\newpage
Es gibt noch einen alternativen L"osungsweg, der weitaus k"urzer (und eleganter) ist:
\begin{eqnarray*}
\langle\vec{x}'|\hat{U}(T,0)|\vec{x}\rangle & = &  \langle\vec{x}'|\exp\left(\frac{i}{\hbar}\frac{\hat{p}^2}{2m}T\right) |\vec{x}\rangle \\ & = & \int d^3p \langle\vec{x}'|\exp\left(\frac{i}{\hbar}\frac{\hat{p}^2}{2m}T\right)|\vec{p}\rangle \langle\vec{p}|\vec{x}\rangle \\ & = & \frac{1}{(2\pi\hbar)^3} \int \! d^3p \, \exp\left(\frac{i}{\hbar}\frac{p^2}{2m}T\right) \exp\left(\frac{i}{\hbar}p(\vec{x}'-\vec{x})\right) \\ & = & \frac{1}{(2\pi\hbar)^3}\left(\frac{2\pi\hbar m}{iT}\right)^\frac{3}{2} \exp\left(\frac{i}{\hbar}\frac{m}{2}\frac{(\vec{x}' - \vec{x})^2}{T}\right)\\
& = & \left( \frac{m}{2\pi i\hbar T}\right)^{\frac{3}{2}} \exp\left(\frac{i}{\hbar}\frac{m}{2}\frac{(\vec{x}' - \vec{x})^2}{T}\right)
\end{eqnarray*}
\end{document}
